% # TODO: Cấu trúc luận văn
Nội dung luận văn bao gồm 3 chương:


\begin{itemize}
    \item \textbf{Chương 1 -- Giới thiệu:} Giới thiệu tổng quan về đề tài, mục tiêu, các phương pháp nghiên cứu, đối tượng và phạm vi nghiên cứu của đề tài.
    Chương này cũng sẽ giới thiệu sơ lược về khả năng ứng dụng của xử lý ngôn ngữ tự nhiên vào bài toán thực tế - cụ thể là hệ thống hỏi đáp về thủ
    tục hành chính ở địa bàn Cần Thơ.
    \item \textbf{Chương 2 -- Nội dung:} gồm 3 phần:
          \begin{itemize}
              \item \textbf{Phần 1 -- Cơ sở lý thuyết}: Phần này sẽ đi sâu vào các cơ sở lý thuyết của các giải pháp đã được áp dụng vào hệ thống hỏi đáp thủ tục hành chính.
              Các khái niệm và lý thuyết sẽ được đề cập đến bao gồm các phương pháp nhúng từ, mô hình Seq2Seq, kiến trúc Transformer, mô hình BERT, 
              độ tương đồng ngữ nghĩa Cosine, mạng sinh đôi (Siamese network), mô hình SBERT, RAG và các phương pháp có liên quan, cách đánh giá kết quả truy vấn thông tin.
              \item \textbf{Phần 2 -- Phương pháp thực hiện}: Phần này sẽ miêu tả cách thực hiện hệ thống hỏi đáp về thủ tục hành chính dựa trên RAG, cách xây dựng giải pháp và huấn luyện mô hình.
              \item \textbf{Phần 3 -- Kết quả thực nghiệm}: Phần này sẽ mô tả cách đánh giá các giải pháp, tiến hành thực nghiệm trên mô hình, kết quả đạt được sau khi
              đánh giá mô hình. 
          \end{itemize}
    \item \textbf{Chương 3 -- Kết luận:}: Phần này sẽ tổng kết kết quả đạt được của đề tài, nhận định về kết quả và một số hướng phát triển.
\end{itemize}